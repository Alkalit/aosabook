\begin{aosachapter}{Eclipse}{s:eclipse}{Kim Moir}

Implementing software modularity is a notoriously difficult task.
Interoperability with a large code base written by a diverse community
is also difficult to manage. At Eclipse, we have managed to succeed on
both counts. In June 2010, the Eclipse Foundation made available its
Helios coordinated release, with over 39 projects and 490 committers
from over 40 companies working together to build upon the
functionality of the base platform. What was the original
architectural vision for Eclipse?  How did it evolve? How does the
architecture of an application serve to encourage community engagement
and growth? Let's go back to the beginning.

On November 7, 2001, an open source project called Eclipse 1.0 was
released. At the time, Eclipse was described as ``an integrated
development environment (IDE) for anything and nothing in
particular.''  This description was purposely generic because the
architectural vision was not just another set of tools, but a
framework; a framework that was modular and scalable. Eclipse provided
a component-based platform that could serve as the foundation for
building tools for developers. This extensible architecture encouraged
the community to build upon a core platform and extend it beyond the
limits of the original vision. Eclipse started as a platform and the
Eclipse SDK was the proof-of-concept product.  The Eclipse SDK allowed
the developers to self-host and use the Eclipse SDK itself to build
newer versions of Eclipse.

The stereotypical image of an open source developer is that of an
altruistic person toiling late into night fixing bugs and implementing
fantastic new features to address their own personal interests. In
contrast, if you look back at the early history of the Eclipse
project, some of the initial code that was donated was based on
VisualAge for Java, developed by IBM\@. The first committers who worked
on this open source project were employees of an IBM subsidiary
called Object Technology International (OTI). These committers were
paid to work full time on the open source project, to answer questions
on newsgroups, address bugs, and implement new features. A consortium
of interested software vendors was formed to expand this open tooling
effort. The initial members of the Eclipse consortium were Borland,
IBM, Merant, QNX Software Systems, Rational Software, RedHat, SuSE,
and TogetherSoft.

By investing in this effort, these companies would have the expertise
to ship commercial products based on Eclipse. This is similar to
investments that corporations make in contributing to the Linux kernel
because it is in their self-interest to have employees improving the
open source software that underlies their commercial offerings. In
early 2004, the Eclipse Foundation was formed to manage and expand the
growing Eclipse community.  This not-for-profit foundation was funded
by corporate membership dues and is governed by a board of
directors. Today, the diversity of the Eclipse community has expanded
to include over 170 member companies and almost 1000 committers.

Originally, people knew ``Eclipse'' as the SDK only but today it is
much more.  In July 2010, there were 250 diverse projects under
development at eclipse.org. There's tooling to support developing with
C/C++, PHP, web services, model driven development, build tooling and
many more. Each of these projects is included in a top-level project
(TLP) which is managed by a project management committee (PMC)
consisting of senior members of the project nominated for the
responsibility of setting technical direction and release goals. In
the interests of brevity, the scope of this chapter will be limited to
the evolution of the architecture of the Eclipse SDK within 
Eclipse\footnote{\url{http://www.eclipse.org}} and Runtime 
Equinox\footnote{\url{http://www.eclipse.org/equinox}} projects. Since Eclipse has
long history, I'll be focusing on early Eclipse, as well as the 3.0,
3.4 and 4.0 releases.

\begin{aosasect1}{Early Eclipse}

At the beginning of the 21st century, there were many tools for
software developers, but few of them worked together. Eclipse sought
to provide an open source platform for the creation of interoperable
tools for application developers. This would allow
developers to focus on writing new tools, instead of
writing code to deal with infrastructure issues like interacting
with the filesystem, providing software updates, and connecting to
source code repositories. Eclipse is perhaps most famous for the Java
Development Tools (JDT). The intent was that these exemplary Java
development tools would serve as an example for people interested in
providing tooling for other languages.

Before we delve into the architecture of Eclipse, let's look at what
the Eclipse SDK looks like to a developer.  Upon starting Eclipse and
selecting the workbench, you'll be presented with the Java
perspective. A perspective organizes the views and editors that are
specific to the tooling that is currently in use.

\aosafigure{../images/eclipse/javaperspective.png}{Java Perspective}{fig.ecl.jpersp}

Early versions of the Eclipse SDK architecture had three major
elements, which corresponded to three major sub-projects: the
Platform, the JDT (Java Development Tools) and the PDE (Plug-in
Development Environment).

\begin{aosasect2}{Platform}

The Eclipse platform is written using Java and a Java VM is required
to run it. It is built from small units of functionality called
plugins.  Plugins are the basis of the Eclipse component model. A
plugin is essentially a JAR file with a manifest which describes
itself, its dependencies, and how it can be utilized, or
extended. This manifest information was initially stored in
a \code{plug-in.xml} file which resides in the root of the plugin
directory.  The Java development tools provided plugins for
developing in Java. The Plug-in Development Environment (PDE) provides
tooling for developing plugins to extend Eclipse.  Eclipse plugins
are written in Java but could also contain non-code contributions such
as HTML files for online documentation. Each plugin has its own class
loader.  Plugins can express dependencies on other plugins by the
use of \code{requires} statements in the \code{plugin.xml}. Looking at the
\code{plugin.xml} for the \code{org.eclipse.ui} plugin you can see its name
and version specified, as well as the dependencies it needs to import
from other plugins.

\begin{verbatim}
<?xml version="1.0" encoding="UTF-8"?>
<plugin
   id="org.eclipse.ui"
   name="%Plugin.name"
   version="2.1.1"
   provider-name="%Plugin.providerName"
   class="org.eclipse.ui.internal.UIPlugin">

   <runtime>
      <library name="ui.jar">
         <export name="*"/>
         <packages prefixes="org.eclipse.ui"/>
      </library>
   </runtime>
   <requires>
      <import plugin="org.apache.xerces"/>
      <import plugin="org.eclipse.core.resources"/>
      <import plugin="org.eclipse.update.core"/>
      :       :        :
      <import plugin="org.eclipse.text" export="true"/>
      <import plugin="org.eclipse.ui.workbench.texteditor" export="true"/>
      <import plugin="org.eclipse.ui.editors" export="true"/>
   </requires>
</plugin>
\end{verbatim}

In order to encourage people to build upon the Eclipse platform, there
needs to be a mechanism to make a contribution to the platform, and
for the platform to accept this contribution. This is achieved through
the use of extensions and extension points, another element of the
Eclipse component model. The export identifies the interfaces that you
expect others to use when writing their extensions, which limits the
classes that are available outside your plugin to the ones that are
exported. It also provides additional limitations on the resources
that are available outside the plugin, as opposed to making all
public methods or classes available to consumers.  Exported plugins
are considered public API\@. All others are considered private
implementation details. To write a plugin that would contribute a
menu item to the Eclipse toolbar, you can use the \code{actionSets}
extension point in the \code{org.eclipse.ui} plugin.

\scriptsize
\begin{verbatim}
<extension-point id="actionSets" name="%ExtPoint.actionSets"
                 schema="schema/actionSets.exsd"/>
<extension-point id="commands" name="%ExtPoint.commands"
                 schema="schema/commands.exsd"/>
<extension-point id="contexts" name="%ExtPoint.contexts"
                 schema="schema/contexts.exsd"/>
<extension-point id="decorators" name="%ExtPoint.decorators"
                 schema="schema/decorators.exsd"/>
<extension-point id="dropActions" name="%ExtPoint.dropActions"
                 schema="schema/dropActions.exsd"/> =
\end{verbatim}
\normalsize

Your plugin's extension to contribute a menu item to the
\code{org.eclipse.ui.actionSet} extension point would look like:

\begin{verbatim}
<?xml version="1.0" encoding="UTF-8"?>
<plugin
   id="com.example.helloworld"
   name="com.example.helloworld"
   version="1.0.0">
   <runtime>
      <library name="helloworld.jar"/>
   </runtime>
   <requires>
      <import plugin="org.eclipse.ui"/>
   </requires>
   <extension
         point="org.eclipse.ui.actionSets">
      <actionSet
            label="Example Action Set"
            visible="true"
            id="org.eclipse.helloworld.actionSet">
         <menu
               label="Example &Menu"
               id="exampleMenu">
            <separator
                  name="exampleGroup">
            </separator>
         </menu>
         <action
               label="&Example Action"
               icon="icons/example.gif"
               tooltip="Hello, Eclipse world"
               class="com.example.helloworld.actions.ExampleAction"
               menubarPath="exampleMenu/exampleGroup"
               toolbarPath="exampleGroup"
               id="org.eclipse.helloworld.actions.ExampleAction">
         </action>
      </actionSet>
   </extension>
</plugin>
\end{verbatim}

When Eclipse is started, the runtime platform scans the manifests of
the plugins in your install, and builds a plugin registry that is
stored in memory.  Extension points and the corresponding extensions
are mapped by name. The resulting plugin registry can be referenced
from the API provided by the Eclipse platform.  The registry is cached to
disk so that this information can be reloaded the next time Eclipse is
restarted. All plugins are discovered upon startup to populate the
registry but they are not activated (classes loaded) until the code is
actually used. This approach is called lazy activation. The
performance impact of adding additional bundles into your install is
reduced by not actually loading the classes associated with the
plugins until they are needed. For instance, the plugin that
contributes to the org.eclipse.ui.actionSet extension point wouldn't
be activated until the user selected the new menu item in the toolbar.

\aosafigure[300pt]{../images/eclipse/examplemenu.png}{Example Menu}{fig.ecl.menu}

The code that generates this menu item looks like this:

\begin{verbatim}
package com.example.helloworld.actions;

import org.eclipse.jface.action.IAction;
import org.eclipse.jface.viewers.ISelection;
import org.eclipse.ui.IWorkbenchWindow;
import org.eclipse.ui.IWorkbenchWindowActionDelegate;
import org.eclipse.jface.dialogs.MessageDialog;

public class ExampleAction implements IWorkbenchWindowActionDelegate {
    private IWorkbenchWindow window;

    public ExampleAction() {
    }

    public void run(IAction action) {
        MessageDialog.openInformation(
            window.getShell(),
            "org.eclipse.helloworld",
            "Hello, Eclipse architecture world");
    }

    public void selectionChanged(IAction action, ISelection selection) {
    }

    public void dispose() {
    }

    public void init(IWorkbenchWindow window) {
        this.window = window;
    }
}
\end{verbatim}

Once the user selects the new item in the toolbar, the extension
registry is queried by the plugin implementing the extension point.
The plugin supplying the extension instantiates the contribution, and
loads the plugin. Once the plugin is activated, the \code{ExampleAction}
constructor in our example is run, and then initializes a \code{Workbench}
action delegate. Since the selection in the workbench has changed and
the delegate has been created, the action can change. The message
dialog opens with the message ``Hello, Eclipse architecture world''.

This extensible architecture was one of the keys to the successful
growth of the Eclipse ecosystem. Companies or individuals could
develop new plugins, and either release them as open source or sell
them commercially.

One of the most important concepts about Eclipse is that
\emph{everything is a plugin}. Whether the plugin is included in the
Eclipse platform, or you write it yourself, plugins are all first
class components of the assembled
application. \aosafigref{fig.ecl.plat} shows clusters of related
functionality contributed by plugins in early versions of Eclipse.

\aosafigure[250pt]{../images/eclipse/platform.png}{Early Eclipse Architecture}{fig.ecl.plat}

The workbench is the most familiar UI element to users of the Eclipse
platform, as it provides the structures that organize how Eclipse
appears to the user on the desktop. The workbench consists of
perspectives, views, and editors.  Editors are associated with file
types so the correct editor is launched when a file is opened. An
example of a view is the ``problems'' view that indicates errors or
warnings in your Java code. Together, editors and views form a
perspective which presents the tooling to the user in an organized
fashion.

The Eclipse workbench is built on the Standard Widget Toolkit (SWT)
and JFace, and SWT deserves a bit of exploration.  Widget toolkits are
generally classified as either native or emulated.  A native widget
toolkit uses operating system calls to build user interface components
such as lists and push buttons.  Interaction with components is
handled by the operating system. An emulated widget toolkit implements
components outside of the operating system, handling mouse and
keyboard, drawing, focus and other widget functionality itself, rather
than deferring to the operating system.  Both designs have different
strengths and weaknesses.

Native widget toolkits are ``pixel perfect.'' Their widgets look and
feel like their counterparts in other applications on the
desktop. Operating system vendors constantly change the look and feel
of their widgets and add new features. Native widget toolkits get
these updates for free.  Unfortunately, native toolkits are difficult
to implement because their underlying operating system widget
implementations are vastly different, leading to inconsistencies and
programs that are not portable.

Emulated widget toolkits either provide their own look and feel, or
try to draw and behave like the operating system. Their great strength
over native toolkits is flexibility (although modern native widget
toolkits such as Windows Presentation Framework (WPF) are equally as
flexible). Because the code to implement a widget is part of the
toolkit rather than embedded in the operating system, a widget can be
made to draw and behave in any manner. Programs that use emulated
widget toolkits are highly portable.  Early emulated widget toolkits
had a bad reputation. They were often slow and did a poor job of
emulating the operating system, making them look out of place on the
desktop.  In particular, Smalltalk-80 programs at the time were easy
to recognize due to their use of emulated widgets. Users were aware
that they were running a ``Smalltalk program'' and this hurt
acceptance of applications written in Smalltalk.

Unlike other computer languages such as C and C++, the first versions
of Java came with a native widget toolkit library called the Abstract
Window Toolkit (AWT). AWT was considered to be limited, buggy and
inconsistent and was widely decried. At Sun and elsewhere, in part
because of experience with AWT, a native widget toolkit that was
portable and performant was considered to be unworkable. The solution
was Swing, a full-featured emulated widget toolkit.

Around 1999, OTI was using Java to implement a product called
VisualAge Micro Edition. The first version of VisualAge Micro Edition
used Swing and OTI's experience with Swing was not positive. Early
versions of Swing were buggy, had timing and memory issues and the
hardware at the time was not powerful enough to give acceptable
performance. OTI had successfully built a native widget toolkit for
Smalltalk-80 and other Smalltalk implementations to gain acceptance of
Smalltalk.  This experience was used to build the first version of
SWT\@. VisualAge Micro Edition and SWT were a success and SWT was the
natural choice when work began on Eclipse.  The use of SWT over Swing
in Eclipse split the Java community. Some saw conspiracies, but
Eclipse was a success and the use of SWT differentiated it from other
Java programs.  Eclipse was performant, pixel perfect and the general
sentiment was, ``I can't believe it's a Java program.''

Early Eclipse SDKs ran on Linux and Windows. In 2010, there is support
for over a dozen platforms. A developer can write an application for
one platform, and deploy it to multiple platforms. Developing a new
widget toolkit for Java was a contentious issue within the Java
community at the time, but the Eclipse committers felt that it was
worth the effort to provide the best native experience on the
desktop. This assertion applies today, and there are millions of lines
of code that depend on SWT.

JFace is a layer on top of SWT that provides tools for common UI
programming tasks, such as frameworks for preferences and
wizards. Like SWT, it was designed to work with many windowing
systems. However, it is pure Java code and doesn't contain any native
platform code.

The platform also provided an integrated help system based upon small
units of information called topics. A topic consists of a label and a
reference to its location.  The location can be an HTML documentation
file, or an XML document describing additional links.  Topics are
grouped together in table of contents (TOCs).  Consider the topics as
the leaves, and TOCs as the branches of organization. To add help
content to your application, you can contribute to the
\code{org.eclipse.help.toc} extension point, as the
\code{org.eclipse.platform.doc.isv} \code{plugin.xml} does below.

\begin{verbatim}
<?xml version="1.0" encoding="UTF-8"?>
<?eclipse version="3.0"?>
<plugin>

<!-- ===================================================================== -->
<!-- Define primary TOC                                                    -->
<!-- ===================================================================== -->
   <extension
         point="org.eclipse.help.toc">
      <toc
            file="toc.xml"
            primary="true">
      </toc>
      <index path="index"/>
   </extension>
<!-- ===================================================================== -->
<!-- Define TOCs                                                           -->
<!-- ===================================================================== -->
   <extension
         point="org.eclipse.help.toc">
      <toc
            file="topics_Guide.xml">
      </toc>
      <toc
            file="topics_Reference.xml">
      </toc>
      <toc
            file="topics_Porting.xml">
      </toc>
      <toc
            file="topics_Questions.xml">
      </toc>
      <toc
            file="topics_Samples.xml">
      </toc>
   </extension>
\end{verbatim}

Apache Lucene is used to index and search the online help content. In
early versions of Eclipse, online help was served as a Tomcat web
application. Additionally, by providing help within Eclipse itself, you
can also use the subset of help plugins to provide a standalone help
server.\footnote{For example: \url{http://help.eclipse.org}.}

Eclipse also provides team support to interact with a source code
repository, create patches and other common tasks.  The workspace
provided collection of files and metadata that stored your work on the
filesystem. There was also a debugger to trace problems in the
Java code, as well as a framework for building language specific
debuggers.

One of the goals of the Eclipse project was to encourage open source
and commercial consumers of this technology to extend the platform to
meet their needs, and one way to encourage this adoption is to provide a
stable API\@. An API can be thought of as a technical contract
specifying the behavior of your application. It also can be thought
of as a social contract. On the Eclipse project, the mantra is, ``API
is forever''. Thus careful consideration must be given when writing an
API given that it is meant to be used indefinitely. A stable API is
a contract between the client or API consumer and the provider. This
contract ensures that the client can depend on the Eclipse platform to
provide the API for the long term without the need for painful
refactoring on the part of the client.  A good API is also flexible
enough to allow the implementation to evolve.

\end{aosasect2}

\begin{aosasect2}{Java Development Tools (JDT)}

The JDT provides Java editors, wizards, refactoring support, debugger,
compiler and an incremental builder.  The compiler is also used for
content assist, navigation and other editing features. A Java SDK
isn't shipped with Eclipse so it's up to the user to choose which SDK to
install on their desktop. Why did the JDT team write a separate
compiler to compile your Java code within Eclipse?  They had an
initial compiler code contribution from VisualAge Micro Edition. They
planned to build tooling on top of the compiler, so writing the
compiler itself was a logical decision. This approach also allowed the
JDT committers to provide extension points for extending the
compiler. This would be difficult if the compiler was a command line
application provided by a third party.

Writing their own compiler provided a mechanism to provide support for
an incremental builder within the IDE\@. An incremental builder provides
better performance because it only recompiles files that have changed
or their dependencies. How does the incremental builder work? When
you create a Java project within Eclipse, you are creating resources
in the workspace to store your files.  A builder within Eclipse takes
the inputs within your workspace (\code{.java} files), and creates an
output (\code{.class} files).  Through the build state, the builder
knows about the types (classes or interfaces) in the workspace, and
how they reference each other. The build state is provided to the
builder by the compiler each time a source file is compiled. When an
incremental build is invoked, the builder is supplied with a resource
delta, which describes any new, modified or deleted files. Deleted
source files have their corresponding class files deleted. New or
modified types are added to a queue.  The files in the queue are
compiled in sequence and compared with the old class file to determine
if there are structural changes. Structural changes are modifications
to the class that can impact another type that references it. For
example, changing a method signature, or adding or removing a
method. If there are structural changes, all the types that reference
it are also added to the queue.  If the type has changed at all, the
new class file is written to the build output folder.  The build state
is updated with reference information for the compiled type. This
process is repeated for all the types in the queue until empty. If
there are compilation errors, the Java editor will create problem
markers. Over the years, the tooling that JDT provides has expanded
tremendously in concert with new versions of the Java runtime itself.

\end{aosasect2}

\begin{aosasect2}{Plug-in Development Environment (PDE) }

The Plug-in Development Environment (PDE) provided the tooling to
develop, build, deploy and test plugins and other artifacts that are
used to extend the functionality of Eclipse. Since Eclipse plugins
were a new type of artifact in the Java world there wasn't a build
system that could transform the source into plugins.  Thus the PDE
team wrote a component called PDE Build which examined the
dependencies of the plugins and generated Ant scripts to construct
the build artifacts.

\end{aosasect2}

\end{aosasect1}

\begin{aosasect1}{Eclipse 3.0: Runtime, RCP and Robots}

\begin{aosasect2}{Runtime}

Eclipse 3.0 was probably one of the most important Eclipse releases
due to the number of significant changes that occurred during this
release cycle.  In the pre-3.0 Eclipse architecture, the Eclipse component
model consisted of plugins that could interact with each other in two
ways.  First, they could express their dependencies by the use of
the \code{requires} statement in their \code{plugin.xml}.  If plugin
A requires plugin B, plugin A can see all the Java classes and
resources from B, respecting Java class visibility conventions.  Each
plugin had a version, and they could also specify the versions of
their dependencies.  Secondly, the component model provided
\emph{extensions} and \emph{extension points}.  Historically, Eclipse
committers wrote their own runtime for the Eclipse SDK to manage
classloading, plugin dependencies and extensions and extension
points. 

The Equinox project was created as a new incubator project at Eclipse.
The goal of the Equinox project was to replace the Eclipse component
model with one that already existed, as well as provide
support for dynamic plugins. The solutions under consideration
included JMX, Jakarta Avalon and OSGi. JMX was not a fully developed component
model so it was not deemed appropriate. Jakarta Avalon wasn't chosen
because it seemed to be losing momentum as a project. In addition to
the technical requirements, it was also important to consider the
community that supported these technologies.  Would they be willing to
incorporate Eclipse-specific changes? Was it actively developed and
gaining new adopters?  The Equinox team felt that the community around
their final choice of technology was just as important as the
technical considerations.

After researching and evaluating the available alternatives, the
committers selected OSGi. Why OSGi?  It had a semantic versioning
scheme for managing dependencies. It provided a framework for
modularity that the JDK itself lacked. Packages that were available to
other bundles must be explicitly exported, and all others were hidden.
OSGi provided its own classloader so the Equinox team didn't have to
continue to maintain their own. By standardizing on a component model
that had wider adoption outside the Eclipse ecosystem, they felt they
could appeal to a broader community and further drive the adoption of
Eclipse.

The Equinox team felt comfortable that since OSGi already had an
existing and vibrant community, they could work with that community to
help include the functionality that Eclipse required in a component
model.  For instance, at the time, OSGi only supported listing
requirements at a package level, not a plugin level as Eclipse
required. In addition, OSGi did not yet include the concept of
fragments, which were Eclipse's preferred mechanism for supplying
platform or environment specific code to an existing plugin. For
example, fragments provide code for working with Linux and Windows
filesystems as well as fragments which contribute language
translations. Once the decision was made to proceed with OSGi as the
new runtime, the committers needed an open source framework
implementation. They evaluated Oscar, the precursor to Apache Felix,
and the Service Management Framework (SMF) developed by IBM\@. At the
time, Oscar was a research project with limited deployment.  SMF was
ultimately chosen since it was already used in shipping products and
thus was deemed enterprise-ready. The Equinox implementation serves as
the reference implementation of the OSGi specification.

A compatibility layer was also provided so that existing plugins
would still work in a 3.0 install. Asking developers to rewrite their
plugins to accommodate changes in the underlying infrastructure of
Eclipse 3.0 would have stalled the momentum on Eclipse as a tooling
platform. The expectation from Eclipse consumers was that the platform
should just continue to work.

With the switch to OSGi, Eclipse plugins became known as bundles. A
plugin and a bundle are the same thing: They both provide a modular
subset of functionality that describes itself with metadata in a
manifest.  Previously, dependencies, exported packages and the
extensions and extension points were described in \code{plugin.xml}.
With the move to OSGi bundles, the extensions and extension points
continued to be described in \code{plugin.xml} since they are Eclipse
concepts. The remaining information was described in
the \code{META-INF/MANIFEST.MF}, OSGi's version of the bundle
manifest. To support this change, PDE provided a new manifest editor
within Eclipse.  Each bundle has a name and version. The manifest for
\code{the org.eclipse.ui} bundle looks like this:

\begin{verbatim}
Manifest-Version: 1.0
Bundle-ManifestVersion: 2
Bundle-Name: %Plugin.name
Bundle-SymbolicName: org.eclipse.ui; singleton:=true
Bundle-Version: 3.3.0.qualifier
Bundle-ClassPath: .
Bundle-Activator: org.eclipse.ui.internal.UIPlugin
Bundle-Vendor: %Plugin.providerName
Bundle-Localization: plugin
Export-Package: org.eclipse.ui.internal;x-internal:=true
Require-Bundle: org.eclipse.core.runtime;bundle-version="[3.2.0,4.0.0)",
 org.eclipse.swt;bundle-version="[3.3.0,4.0.0)";visibility:=reexport,
 org.eclipse.jface;bundle-version="[3.3.0,4.0.0)";visibility:=reexport,
 org.eclipse.ui.workbench;bundle-version="[3.3.0,4.0.0)";visibility:=reexport,
 org.eclipse.core.expressions;bundle-version="[3.3.0,4.0.0)"
Eclipse-LazyStart: true
Bundle-RequiredExecutionEnvironment: CDC-1.0/Foundation-1.0, J2SE-1.3
\end{verbatim}

As of Eclipse 3.1, the manifest can also specify a bundle required
execution environment (BREE). Execution environments specify the
minimum Java environment required for the bundle to run. The Java
compiler does not understand bundles and OSGi manifests. PDE provides
tooling for developing OSGi bundles. Thus, PDE parses the bundle's
manifest, and generates the classpath for that bundle. If you
specified an execution environment of J2SE-1.4 in your manifest, and
then wrote some code that included generics, you would be advised of
compile errors in your code. This ensures that your code adheres to
the contract you have specified in the manifest.

OSGi provides a modularity framework for Java. The OSGi framework
manages collections of self-describing bundles and manages their
classloading. Each bundle has its own classloader. The classpath
available to a bundle is constructed by examining the dependencies of
the manifest and generating a classpath available to the bundle. OSGi
applications are collections of bundles. In order to fully embrace
modularity, you must be able to express your dependencies in a
reliable format for consumers. Thus the manifest describes exported
packages that are available to clients of this bundle which
corresponds to the public API that was available for consumption.  The
bundle that is consuming that API must have a corresponding import of
the package they are consuming. The manifest also allows you to
express version ranges for your dependencies.  Looking at
the \code{Require-Bundle} heading in the above manifest, you will note
that the \code{org.eclipse.core.runtime} bundle that
\code{org.eclipse.ui} depends on must be at least 3.2.0 and less than
4.0.0.

\aosafigure[200pt]{../images/eclipse/bundlelifecycle.png}{OSGi Bundle Lifecycle}{fig.ecl.blc}

OSGi is a dynamic framework which supports the installation, starting,
stopping, or uninstallation of bundles. As mentioned before, lazy
activation was a core advantage to Eclipse because plugin classes were
not loaded until they were needed. The OSGi bundle lifecycle also
enables this approach.  When you start an OSGi application, the
bundles are in the installed state. If its dependencies are met, the
bundle changes to the resolved state.  Once resolved, the classes
within that bundle can be loaded and run.  The starting state means
that the bundle is being activated according to its activation
policy. Once activated, the bundle is in the active state, it can
acquire required resources and interact with other bundles. A bundle
is in the stopping state when it is executing its activator stop method
to clean up any resources that were opened when it was active.
Finally, a bundle may be uninstalled, which means that it's not
available for use.

As the API evolves, there needs to be a way to signal changes to your
consumers. One approach is to use semantic versioning of your bundles
and version ranges in your manifests to specify the version ranges for
your dependencies.  OSGi uses a four-part versioning naming scheme as
shown in \aosafigref{fig.ecl.ver}.

\aosafigure[200pt]{../images/eclipse/versioning.png}{Versioning Naming Scheme}{fig.ecl.ver}

With the OSGi version numbering scheme, each bundle has a unique
identifier consisting of a name and a four part version number. An id
and version together denote a unique set of bytes to the
consumer. By Eclipse convention, if you're making changes to a bundle,
each segment of the version signifies to the consumer the type of
change being made. Thus, if you want to indicate that you intend to
break API, you increment the first (major) segment.  If you have just
added API, you increment the second (minor) segment. If you fix a
small bug that doesn't impact API, the third (service) segment is
incremented. Finally, the fourth or qualifier segment is incremented
to indicate a build id source control repository tag.

In addition to expressing the fixed dependencies between bundles,
there is also a mechanism within OSGi called services which provides
further decoupling between bundles. Services are objects with a set of
properties that are registered with the OSGi service registry. Unlike
extensions, which are registered in the extension registry when Eclipse
scans bundles during startup, services are registered
dynamically. A bundle that is consuming a service needs to import the
package defining the service contract, and the framework determines
the service implementation from the service registry.

Like a main method in a Java class file, there is a specific
application defined to start Eclipse. Eclipse applications are defined
using extensions. For instance, the application to start the Eclipse
IDE itself is \code{org.eclipse.ui.ide.workbench} which is defined in the
\code{org.eclipse.ui.ide.application} bundle.

\begin{verbatim}
<plugin>
    <extension
         id="org.eclipse.ui.ide.workbench"
         point="org.eclipse.core.runtime.applications">
      <application>
         <run
               class="org.eclipse.ui.internal.ide.application.IDEApplication">
         </run>
      </application>
  </extension>
</plugin>
\end{verbatim}

There are many applications provided by Eclipse such as those to run
standalone help servers, Ant tasks, and JUnit tests.

\end{aosasect2}

\begin{aosasect2}{Rich Client Platform (RCP)}

One of the most interesting things about working in an open source
community is that people use the software in totally unexpected ways.
The original intent of Eclipse was to provide a platform and tooling
to create and extend IDEs.  However, in the time leading up to the 3.0
release, bug reports revealed that the community was taking a subset
of the platform bundles and using them to build Rich Client Platform
(RCP) applications, which many people would recognize as Java applications.  Since Eclipse was initially constructed with an
IDE-centric focus, there had to be some refactoring of the bundles to
allow this use case to be more easily adopted by the user
community. RCP applications didn't require all the functionality in
the IDE, so several bundles were split into smaller ones that could be
consumed by the community for building RCP applications.  

Examples of RCP applications in the wild include the use of RCP to monitor the
Mars Rover robots developed by NASA at the Jet Propulsion Laboratory,
Bioclipse for data visualization of bioinformatics and Dutch Railway
for monitoring train performance.  The common thread that ran through
many of these applications was that these teams decided that they
could take the utility provided by the RCP platform and concentrate on
building their specialized tools on top of it. They could save
development time and money by focusing on building their tools on a
platform with a stable API that guaranteed that their technology
choice would have long term support.

\aosafigure[300pt]{../images/eclipse/rcp.png}{Eclipse 3.0 Architecture}{fig.ecl.rcp}

Looking at the 3.0 architecture in \aosafigref{fig.ecl.rcp}, you will
note that the Eclipse Runtime still exists to provide the application
model and extension registry.  Managing the dependencies between
components, the plugin model is now managed by OSGi. In addition
to continuing to be able to extend Eclipse for their own IDEs,
consumers can also build upon the RCP application framework for more
generic applications.

\end{aosasect2}

\end{aosasect1}

\begin{aosasect1}{Eclipse 3.4}

The ability to easily update an application to a new version and add
new content is taken for granted. In Firefox it happens seamlessly.
For Eclipse it hasn't been so easy. Update Manager was the original
mechanism that was used to add new content to the Eclipse install or
update to a new version.

To understand what changes during an update or install operation,
it's necessary to understand what Eclipse means by ``features''.
A feature is a PDE artifact that defines a set
of bundles that are packaged together in a format that can be built or
installed. Features can also include other features. (See
\aosafigref{fig.ecl.feat}.)

\aosafigure[250pt]{../images/eclipse/eclipse33features.png}{Eclipse 3.3 SDK Feature Hierarchy}{fig.ecl.feat}

If you wished to update your Eclipse install to a new build that only
incorporated one new bundle, the entire feature had to be updated
since this was the coarse grained mechanism that was used by update
manager.  Updating a feature to fix a single bundle is inefficient.

There are PDE wizards to create features, and build them in your
workspace.  The \code{feature.xml} file defines the bundles included
in the feature, and some simple properties of the bundles. A feature,
like a bundle, has a name and a version. Features can include other
features, and specify version ranges for the features they
include. The bundles that are included in a feature are listed, along
with specific properties. For instance, you can see that the
\code{org.eclipse.launcher.gtk.linux.x86\_64} fragment specifies the operating
system (\code{os}), windowing system (\code{ws}) and architecture
(\code{arch}) where it should be used. Thus upgrading to a new
release, this fragment would only be installed on this platform. These
platform filters are included in the OSGi manifest of this bundle.

\begin{verbatim}
<?xml version="1.0" encoding="UTF-8"?>
<feature
      id="org.eclipse.rcp"
      label="%featureName"
      version="3.7.0.qualifier"
      provider-name="%providerName"
      plugin="org.eclipse.rcp"
      image="eclipse_update_120.jpg">

   <description>
      %description
   </description>

   <copyright>
      %copyright
   </copyright>

   <license url="%licenseURL">
      %license
   </license>
 
   <plugin
         id="org.eclipse.equinox.launcher"
         download-size="0"
         install-size="0"
         version="0.0.0"
         unpack="false"/>

   <plugin
         id="org.eclipse.equinox.launcher.gtk.linux.x86_64"
         os="linux"
         ws="gtk"
         arch="x86_64"
         download-size="0"
         install-size="0"
         version="0.0.0"
         fragment="true"/>
\end{verbatim}

An Eclipse application consists of more than just features and
bundles.  There are platform specific executables to start Eclipse
itself, license files, and platform specific libraries, as shown in
this list of files included in the Eclipse application.

\begin{verbatim}
com.ibm.icu
org.eclipse.core.commands
org.eclipse.core.conttenttype
org.eclipse.core.databinding
org.eclipse.core.databinding.beans
org.eclipse.core.expressions
org.eclipse.core.jobs
org.eclipse.core.runtime
org.eclipse.core.runtime.compatibility.auth
org.eclipse.equinox.common
org.eclipse.equinox.launcher
org.eclipse.equinox.launcher.carbon.macosx
org.eclipse.equinox.launcher.gtk.linux.ppc
org.eclipse.equinox.launcher.gtk.linux.s390
org.eclipse.equinox.launcher.gtk.linux.s390x
org.eclipse.equinox.launcher.gtk.linux.x86
org.eclipse.equinox.launcher.gtk.linux.x86_64
\end{verbatim}

% \aosafigureTop[200pt]{../images/eclipse/rcpfeature33.png}{Files Included in Eclipse Application}{fig.ecl.inst}

These files couldn't be updated via
update manager, because again, it only dealt with features.
Since many of these files were updated every major release, this meant
that users had to download a new zip each time there was a new release
instead of updating their existing install. This wasn't acceptable to
the Eclipse community. PDE provided support for product files, which
specified all the files needed to build an Eclipse RCP
application. However, update manager didn't have a mechanism to
provision these files into your install which was very frustrating for
users and product developers alike. In March 2008, p2 was released
into the SDK as the new provisioning solution.  In the interest of
backward compatibility, Update Manager was still available for use,
but p2 was enabled by default.

\begin{aosasect2}{p2 Concepts}

Equinox p2 is all about installation units (IU). An IU is a
description of the name and id of the artifact you are
installing. This metadata also describes the capabilities of the
artifact (what is provided) and its requirements (its
dependencies). Metadata can also express applicability filters if an
artifact is only applicable to a certain environment. For instance,
the org.eclipse.swt.gtk.linux.x86 fragment is only applicable if
you're installing on a Linux gtk x86 machine. Fundamentally, metadata
is an expression of the information in the bundle's
manifest. Artifacts are simply the binary bits being installed. A
separation of concerns is achieved by separating the metadata and the
artifacts that they describe. A p2 repository consists of both
metadata and artifact repositories.

\aosafigure{../images/eclipse/p2.png}{P2 Concepts}{fig.ecl.p2}

A profile is a list of IUs in your install. For instance, your Eclipse
SDK has a profile that describes your current install. From within
Eclipse, you can request an update to a newer version of the build
which will create a new profile with a different set of IUs. A profile
also provides a list of properties associated with the installation,
such as the operating system, windowing system, and architecture
parameters. Profiles also store the installation directory and the
location. Profiles are held by a profile registry, which can store
multiple profiles. The director is responsible for invoking
provisioning operations. It works with the planner and the engine. The
planner examines the existing profile, and determines the operations
that must occur to transform the install into its new state. The
engine is responsible for carrying out the actual provisioning
operations and installing the new artifacts on disk.  Touchpoints are
part of the engine that work with the runtime implementation of the
system being installed. For instance, for the Eclipse SDK, there is an
Eclipse touchpoint which knows how to install bundles. For a Linux
system where Eclipse is installed from RPM binaries, the engine would
deal with an RPM touchpoint. Also, p2 can perform installs in-process
or outside in a separate process, such as a build.

There were many benefits to the new p2 provisioning system. Eclipse
install artifacts could be updated from release to release. Since
previous profiles were stored on disk, there was also a way to revert
to a previous Eclipse install. Additionally, given a profile and a
repository, you could recreate the Eclipse install of a user that was
reporting a bug to try to reproduce the problem on your own desktop.
Provisioning with p2 provided a way to update and install more than
just the Eclipse SDK, it was a platform that applied to RCP and OSGi
use cases as well.  The Equinox team also worked with the members of
another Eclipse project, the Eclipse Communication Framework (ECF) to
provide reliable transport for consuming artifacts and metadata in p2
repositories.

There were many spirited discussions within the Eclipse community when p2
was released into the SDK\@. Since update manager was a less than
optimal solution for provisioning your Eclipse install, Eclipse
consumers had the habit of unzipping bundles into their install and
restarting Eclipse. This approach resolves your bundles on a best
effort basis. It also meant that any conflicts in your install were
being resolved at runtime, not install time.  Constraints should be
resolved at install time, not run time. However, users were often
oblivious to these issues and assumed since the bundles existed on
disk, they were working.  Previously, the update sites that Eclipse
provided were a simple directory consisting of JARred bundles and
features.  A simple \code{site.xml} file provided the names of the
features that were available to be consumed in the site.  With the
advent of p2, the metadata that was provided in the p2 repositories
was much more complex. To create metadata, the build process needed to
be tweaked to either generate metadata at build time or run a
generator task over the existing bundles. Initially, there was a lack
of documentation available describing how to make these changes.  As
well, as is always the case, exposing new technology to a wider
audience exposed unexpected bugs that had to be addressed. However, by
writing more documentation and working long hours to address these
bugs, the Equinox team was able to address these concerns and now p2
is the underlying provision engine behind many commercial
offerings. As well, the Eclipse Foundation ships its coordinated
release every year using a p2 aggregate repository of all the
contributing projects.

\end{aosasect2}

\end{aosasect1}

\begin{aosasect1}{Eclipse 4.0}

Architecture must continually be examined to evaluate if it is still
appropriate. Is it able to incorporate new technology? Does it
encourage growth of the community?  Is it easy to attract new
contributors?  In late 2007, the Eclipse project committers decided
that the answers to these questions were no and they embarked on
designing a new vision for Eclipse. At the same time, they realized
that there were thousands of Eclipse applications that depended on the
existing API\@.  An incubator technology project was created in late
2008 with three specific goals: simplify the Eclipse programming
model, attract new committers and enable the platform to take
advantage of new web-based technologies while providing an open
architecture.

\aosafigure{../images/eclipse/e4.png}{Eclipse 4.0 SDK Early Adopter Release}{fig.ecl.e4}

Eclipse 4.0 was first released in July 2010 for early adopters to
provide feedback. It consisted of a combination of SDK bundles that
were part of the 3.6 release, and new bundles that graduated from the
technology project.  Like 3.0, there was a compatibility layer so that
existing bundles could work with the new release. As always, there was
the caveat that consumers needed to be using the public API in order
to be assured of that compatibility.  There was no such guarantee if
your bundle used internal code. The 4.0 release provided the Eclipse 4
Application Platform which provided the following features.

\begin{aosasect2}{Model Workbench}

In 4.0, a model workbench is generated using the Eclipse Modeling
Framework (EMFgc). There is a separation of concerns between the model
and the rendering of the view, since the renderer talks to the model
and then generates the SWT code. The default is to use the SWT
renderers, but other solutions are possible.  If you create an example
4.x application, an XMI file will be created for the default workbench
model. The model can be modified and the workbench will be instantly
updated to reflect the changes in the model. \aosafigref{fig.ecl.mwb}
is an example of a model generated for an example 4.x application.

\aosafigure[350pt]{../images/eclipse/modelledworkbench.pdf}{Model Generated for Example 4.x Application}{fig.ecl.mwb}

\end{aosasect2}

\begin{aosasect2}{Cascading Style Sheets Styling}

Eclipse was released in 2001, before the era of rich Internet
applications that could be skinned via CSS to provide a different look
and feel. Eclipse 4.0 provides the ability to use stylesheets to
easily change the look and feel of the Eclipse application. The
default CSS stylesheets can be found in the \code{css} folder of the
\code{org.eclipse.platform} bundle.

\end{aosasect2}

\begin{aosasect2}{Dependency Injection}

Both the Eclipse extensions registry and OSGi services are examples of service
programming models. By convention, a service programming model contains service
producers and consumers. The broker is responsible for managing the
relationship between producers and consumers.

\aosafigure[200pt]{../images/eclipse/producerconsumer.png}{Relationship Between Producers and Consumers}{fig.ecl.prc}

\pagebreak

Traditionally, in Eclipse 3.4.x applications, the consumer needed
to know the location of the implementation, and to
understand inheritance within the framework to
consume services. The consumer code was therefore less reusable
because people couldn't override which implementation the consumer
receives. For example, if you wanted to update the message on the
status line in Eclipse 3.x, the code would look like:

\begin{verbatim}
getViewSite().getActionBars().getStatusLineManager().setMessage(msg);
\end{verbatim}

Eclipse 3.6 is built from components, but many of these components are
too tightly coupled. To assemble applications of more loosely coupled
components, Eclipse 4.0 uses dependency injection to provide services
to clients. Dependency injection in Eclipse 4.x is through the use of
a custom framework that uses the the concept of a context that serves
as a generic mechanism to locate services for consumers.  The context
exists between the application and the framework. Contexts are
hierarchical. If a context has a request that cannot be satisfied,
it will delegate the request to the parent context. The Eclipse
context, called \code{IEclipseContext}, stores the available services and
provides OSGi services lookup.  Basically, the context is similar to a
Java map in that it provides a mapping of a name or class to an
object.  The context handles model elements and services.  Every
element of the model, will have a context. Services are published in
4.x by means of the OSGi service mechanism.

\aosafigure[200pt]{../images/eclipse/context.png}{Service Broker Context}{fig.ecl.con}

Producers add services and objects to the context which stores them.
Services are injected into consumer objects by the context. The
consumer declares what it wants, and the context determines how to
satisfy this request. This approach has made consuming dynamic service
easier. In Eclipse 3.x, a consumer had to attach listeners to be
notified when services were available or unavailable. With Eclipse
4.x, once a context has been injected into a consumer object, any
change is automatically delivered to that object again. In other
words, dependency injection occurs again. The consumer indicates that
it will use the context by the use of Java 5 annotations which adhere
to the JSR~330 standard, such as \code{@inject}, as well as some
custom Eclipse annotations. Constructor, method, and field injection
are supported.  The 4.x runtime scans the objects for these
annotations. The action that is performed depends on the annotation
that's found.

This separation of concerns between context and application allows for
better reuse of components, and absolves the consumer from
understanding the implementation.  In 4.x, the code to update the
status line would look like this:

\begin{verbatim}
@Inject
IStatusLineManager statusLine;
...
statusLine.setMessage(msg);
\end{verbatim}

\end{aosasect2}

\begin{aosasect2}{Application Services}

One of the main goals in Eclipse 4.0 was to simplify the API for
consumers so that it was easy to implement common services. The list
of simple services came to be known as ``the twenty things'' and are
known as the Eclipse Application services. The goal is to offer
standalone APIs that clients can use without having to have a deep
understanding of all the APIs available. They are structured as
individual services so that they can also be used in other languages
other than Java, such as Javascript.  For example, there is an API to
access the application model, to read and modify preferences and
report errors and warnings.

\end{aosasect2}

\end{aosasect1}

\begin{aosasect1}{Conclusion}

The component-based architecture of Eclipse has evolved to incorporate
new technology while maintaining backward compatibility.  This has
been costly, but the reward is the growth of the Eclipse community
because of the trust established that consumers can continue to ship
products based on a stable API.

Eclipse has so many consumers with diverse use cases and our expansive
API became difficult for new consumers to adopt and understand. In
retrospect, we should have kept our API simpler. If 80\% of consumers
only use 20\% of the API, there is a need for simplification which was
one of the reasons that the Eclipse 4.x stream was created.

The wisdom of crowds does reveal interesting use cases, such as
disaggregating the IDE into bundles that could be used to construct
RCP applications.  Conversely, crowds often generate a lot of noise
with requests for edge case scenarios that take a significant amount
of time to implement.

In the early days of the Eclipse project, committers had the luxury of
dedicating significant amounts of time to documentation, examples and
answering community questions. Over time, this responsibility has
shifted to the Eclipse community as a whole.  We could have been
better at providing documentation and use cases to help out the
community, but this has been difficult given the large number of items
planned for every release.  Contrary to the expectation that software
release dates slip, at Eclipse we consistently deliver our releases on
time which allows our consumers to trust that they will be able to do
the same.

By adopting new technology, and reinventing how Eclipse looks and
works, we continue the conversation with our consumers and keep
them engaged in the community. If you're interested in becoming
involved with Eclipse, please visit http://www.eclipse.org.
\end{aosasect1}

\end{aosachapter}
