\begin{aosachapter}{MediaWiki}{s:mediawiki}{Sumana Harihareswara and Guillaume Paumier}

From the start, MediaWiki was developed specifically to be Wikipedia's
software. Developers have worked to facilitate reuse by third-party
users, but Wikipedia's influence and bias have shaped MediaWiki's
architecture throughout its history.

Wikipedia is one of the top ten websites in the world, currently
getting about 400 million unique visitors a month. It gets over
100,000 hits per second. Wikipedia isn't commercially supported by
ads; it is entirely supported by a non-profit organization, the
Wikimedia Foundation, which relies on donations as its primary funding
model. This means that MediaWiki must not only run a top-ten website,
but also do so on a shoestring budget. To meet these demands,
MediaWiki has a heavy bias towards performance, caching and
optimization. Expensive features that can't be enabled on Wikipedia
are either reverted or disabled through a configuration variable;
there is an endless balance between performance and features.

The influence of Wikipedia on MediaWiki's architecture isn't limited
to performance. Unlike generic content management systems (CMSes),
MediaWiki was originally written for a very specific purpose:
supporting a community that creates and curates freely reusable
knowledge on an open platform. This means, for example, that MediaWiki
doesn't include regular features found in corporate CMSes, like a
publication workflow or access control lists, but does offer a variety
of tools to handle spam and vandalism.

So, from the start, the needs and actions of a constantly evolving
community of Wikipedia participants have affected MediaWiki's
development, and vice versa. The architecture of MediaWiki has been
driven many times by initiatives started or requested by the
community, such as the creation of Wikimedia Commons, or the Flagged
Revisions feature. Developers made major architectural changes because
the way that MediaWiki was used by Wikipedians made it necessary.

MediaWiki has also gained a solid external user base by being
open source software from the beginning. Third-party reusers know
that, as long as such a high-profile website as Wikipedia uses
MediaWiki, the software will be maintained and improved. MediaWiki
used to be really focused on Wikimedia sites, but efforts have been
made to make it more generic and better accommodate the needs of these
third-party users. For example, MediaWiki now ships with an excellent
web-based installer, making the installation process much less painful
than when everything had to be done via the command line and the
software contained hardcoded paths for Wikipedia.

Still, MediaWiki is and remains Wikipedia's software, and this shows
throughout its history and architecture.

\pagebreak

This chapter is organized as follows:

\begin{aosaitemize}

\item \emph{Historical Overview} gives a short overview of the history
  of MediaWiki, or rather its prehistory, and the circumstances of its
  creation.

\item \emph{MediaWiki Code Base and Practices} explains the choice of
  PHP, the importance and implementation of secure code, and how
  general configuration is handled.

\item \emph{Database and Text Storage} dives into the distributed data
  storage system, and how its structure evolved to accommodate growth.

\item \emph{Requests, Caching and Delivery} follows the execution of a
  web request through the components of MediaWiki it activates. This
  section includes a description of the different caching layers, and
  the asset delivery system.

\item \emph{Languages} details the pervasive internationalization and
  localization system, why it matters, and how it is implemented.

\item \emph{Users} presents how users are represented in the software,
  and how user permissions work.

\item \emph{Content} details how content is structured, formatted and
  processed to generate the final HTML. A subsection focuses on how
  MediaWiki handles media files.

\item \emph{Customizing and Extending MediaWiki} explains how
  JavaScript, CSS, extensions, and skins can be used to customize a
  wiki, and how they modify its appearance and behavior. A subsection
  presents the software's machine-readable web API.

\end{aosaitemize}

\begin{aosasect1}{Historical Overview}

\begin{aosasect2}{Phase I: UseModWiki}

Wikipedia was launched in January 2001. At the time, it was mostly an
experiment to try to boost the production of content for Nupedia, a
free-content, but peer-reviewed, encyclopedia created by Jimmy
Wales. Because it was an experiment, Wikipedia was originally powered
by UseModWiki, an existing GPL wiki engine written in Perl, using
CamelCase and storing all pages in individual text files with no
history of changes made.

It soon appeared that CamelCase wasn't really appropriate for naming
encyclopedia articles. In late January 2001, UseModWiki developer and
Wikipedia participant Clifford Adams added a new feature to
UseModWiki: free links; i.e., the ability to link to pages with a
special syntax (double square brackets), instead of automatic
CamelCase linking. A few weeks later, Wikipedia upgraded to the new
version of UseModWiki supporting free links, and enabled them.

While this initial phase isn't about MediaWiki per se, it
provides some context and shows that, even before MediaWiki was
created, Wikipedia started to shape the features of the software that
powered it. UseModWiki also influenced some of MediaWiki's features;
for example, its markup language. The Nostalgia
Wikipedia\footnote{\url{http://nostalgia.wikipedia.org}} contains a
complete copy of the Wikipedia database from December 2001, when
Wikipedia still used UseModWiki.

\end{aosasect2}

\begin{aosasect2}{Phase II: The PHP Script}

In 2001, Wikipedia was not yet a top ten website; it was an obscure
project sitting in a dark corner of the Interwebs, unknown to most
search engines, and hosted on a single server. Still, performance was
already an issue, notably because UseModWiki stored its content in a
flat file database. At the time, Wikipedians were worried about being
inundated with traffic following articles in the New York Times,
Slashdot and Wired.

So in summer 2001, Wikipedia participant Magnus Manske (then a
university student) started to work on a dedicated Wikipedia wiki
engine in his free time. He aimed to improve Wikipedia's performance
using a database-driven app, and to develop
Wikipedia-specific features that couldn't be provided by a ``generic''
wiki engine. Written in PHP and MySQL-backed, the new engine was
simply called the ``PHP script'', ``PHP wiki'', ``Wikipedia software'' or
``phase II''.

The PHP script was made available in August 2001, shared on
SourceForge in September, and tested until late 2001. As Wikipedia
suffered from recurring performance issues because of increasing
traffic, the English language Wikipedia eventually switched from
UseModWiki to the PHP script in January 2002. Other language versions
also created in 2001 were slowly upgraded as well, although some of
them would remain powered by UseModWiki until 2004.

As PHP software using a MySQL database, the PHP script was the first
iteration of what would later become MediaWiki. It introduced
many critical features still in use today, like namespaces to organize
content (including talk pages), skins, and special pages (including
maintenance reports, a contributions list and a user watchlist).

\end{aosasect2}

\begin{aosasect2}{Phase III: MediaWiki}

Despite the improvements from the PHP script and database backend,
the combination of increasing traffic, expensive features and limited
hardware continued to cause performance issues on Wikipedia. In 2002,
Lee Daniel Crocker rewrote the code again, calling the new software
``Phase III''\footnote{\url{http://article.gmane.org/gmane.science.linguistics.wikipedia.technical/2794}}. Because the site was experiencing frequent difficulties,
Lee thought there ``wasn't much time to sit down and properly architect
and develop a solution'', so he ``just reorganized the existing
architecture for better performance and hacked all the
code''. Profiling features were added to track down slow functions.

The Phase III software kept the same basic interface, and was designed
to look and behave as much like the Phase II software as possible. A
few new features were also added, like a new file upload system,
side-by-side diffs of content changes, and interwiki links.

Other features were added over 2002, like new maintenance special
pages, and the ``edit on double click'' option. Performance issues
quickly reappeared, though. For example, in November 2002,
administrators had to temporarily disable the ``view count'' and ``site''
statistics which were causing two database writes on every page
view. They would also occasionally switch the site to read-only mode
to maintain the service for readers, and disable expensive maintenance
pages during high-access times because of table locking problems.

In early 2003, developers discussed whether they should properly
re-engineer and re-architect the software from scratch, before the
fire-fighting became unmanageable, or continue to tweak and improve
the existing code base. They chose the latter solution, mostly because
most developers were sufficiently happy with the code base, and
confident enough that further iterative improvements would be enough
to keep up with the growth of the site.

In June 2003, administrators added a second server, the first database
server separate from the web server. (The new machine was also the web
server for non-English Wikipedia sites.) Load-balancing between the
two servers would be set up later that year. Admins also enabled a new
page-caching system that used the file system to cache rendered,
ready-to-output pages for anonymous users.

June 2003 is also when Jimmy Wales created the non-profit Wikimedia 
Foundation
to support Wikipedia and manage its infrastructure and
day-to-day operations. The ``Wikipedia software'' was officially named
``MediaWiki'' in July, as wordplay on the Wikimedia Foundation's
name. What was thought at the time to be a clever pun would confuse
generations of users and developers.

New features were added in July, like the automatically generated
table of contents and the ability to edit page sections, both still
in use today. The first release under the name ``MediaWiki'' happened in
August 2003, concluding the long genesis of an application whose
overall structure would remain fairly stable from there on.

\end{aosasect2}

\end{aosasect1}

\begin{aosasect1}{MediaWiki Code Base and Practices}

\begin{aosasect2}{PHP}

PHP was chosen as the framework for Wikipedia's ``Phase II'' software in
2001; MediaWiki has grown organically since then, and is still
evolving. Most MediaWiki developers are volunteers contributing in
their free time, and there were very few of them in the early years. Some
software design decisions or omissions may seem wrong in retrospect,
but it's hard to criticize the founders for not implementing some
abstraction which is now found to be critical, when the initial code
base was so small, and the time taken to develop it so short.

For example, MediaWiki uses unprefixed class names, which can cause
conflicts when PHP core and PECL (PHP Extension Community Library)
developers add new classes: MediaWiki \code{Namespace} class had to be
renamed to \code{MWNamespace} to be compatible with PHP
5.3. Consistently using a prefix for all classes (e.g., ``\code{MW}'')
would have made it easier to embed MediaWiki inside another
application or library.

Relying on PHP was probably not the best choice for performance, since
it has not benefitted from improvements that some other dynamic
languages have seen. Using Java would have been much better for
performance, and simplified execution scaling for back-end maintenance
tasks. On the other hand, PHP is very popular, which facilitates
recruiting new developers.

Even if MediaWiki still contains ``ugly'' legacy code, major
improvements have been made over the years, and new architectural
elements have been introduced to MediaWiki throughout its
history. They include the \code{Parser}, \code{SpecialPage}, and
\code{Database} classes, the \code{Image} class and the
\code{FileRepo} class hierarchy, ResourceLoader, and the \code{Action}
hierarchy. MediaWiki started without any of these things, but all of
them support features that have been around since the beginning. Many
developers are interested primarily in feature development and
architecture is often left behind, only to catch up later as the cost
of working within an inadequate architecture becomes apparent.

\end{aosasect2}

\begin{aosasect2}{Security}

Because MediaWiki is the platform for high-profile sites such as
Wikipedia, core developers and code reviewers have enforced strict
security rules\footnote{See
  \url{https://www.mediawiki.org/wiki/Security_for_developers} for a
  detailed guide.}. To make it easier to write secure code, MediaWiki
gives developers wrappers around HTML output and database queries to
handle escaping. To sanitize user input, a develop uses the
\code{WebRequest} class, which analyzes data passed in the URL or via
a POSTed form. It removes ``magic quotes'' and slashes, strips illegal input
characters and normalizes Unicode sequences. Cross-site request
forgery (CSRF) is avoided by using tokens, and cross-site scripting
(XSS) by validating inputs and escaping outputs, usually with PHP's
\code{htmlspecialchars()} function. MediaWiki also provides (and uses)
an XHTML sanitizer with the \code{Sanitizer} class, and database
functions that prevent SQL injection.

\end{aosasect2}

\begin{aosasect2}{Configuration}

MediaWiki offers hundreds of configuration settings, stored in global
PHP variables. Their default value is set in
\code{DefaultSettings.php}, and the system administrator can override
them by editing \code{LocalSettings.php}.

MediaWiki used to over-depend on global variables, including for
configuration and context processing. Globals cause serious security
implications with PHP's \code{register\_globals} function (which
MediaWiki hasn't needed since version 1.2). This system also limits
potential abstractions for configuration, and makes it more difficult
to optimize the start-up process. Moreover, the configuration
namespace is shared with variables used for registration and object
context, leading to potential conflicts. From a user perspective,
global configuration variables have also made MediaWiki seem difficult
to configure and maintain. MediaWiki development has been a story of
slowly moving context out of global variables and into
objects. Storing processing context in object member variables allows
those objects to be reused in a much more flexible way.

\end{aosasect2}

\end{aosasect1}

\begin{aosasect1}{Database and Text Storage}

%% I'm going to take out this image because it's huge and I'm 
%% pretty sure it won't print properly, and the chapter can do
%% without it. Maybe keep it for web? --ARB
%% \aosafigure{../images/mediawiki/database-schema.png}{Database schema.}{fig.mediawiki.schema}

MediaWiki has been using a relational database backend since the
Phase II software. The default (and best-supported) database
management system (DBMS) for MediaWiki is MySQL, which is the one that
all Wikimedia sites use, but other DBMSes (such as PostgreSQL, Oracle,
and SQLite) have community-supported implementations. A sysadmin can
choose a DBMS while installing MediaWiki, and MediaWiki provides both
a database abstraction and a query abstraction layer that simplify
database access for developers.

The current layout contains dozens of tables. Many are about the
wiki's content (e.g., \code{page}, \code{revision}, \code{category},
and \code{recentchanges}). Other tables include data about users
(\code{user},\linebreak 
\code{user\_groups}), media files (\code{image},
\code{filearchive}), caching (\code{objectcache}, \code{l10n\_cache}, \linebreak
\code{querycache}) and internal tools (\code{job} for the job queue),
among others\footnote{Complete documentation of the database layout in
  MediaWiki is available at
  \url{https://www.mediawiki.org/wiki/Manual:Database_layout}.}. (See
\aosafigref{fig.mediawiki.restructure}.) Indices
and summary tables are used extensively in MediaWiki, since SQL
queries that scan huge numbers of rows can be very expensive,
particularly on Wikimedia sites. Unindexed queries are usually
discouraged.

The database went through dozens of schema changes over the years, the
most notable being the decoupling of text storage and revision
tracking in MediaWiki 1.5.

\aosafigure[230px]{../images/mediawiki/database-restructure.png}{Main content tables in MediaWiki 1.4 and 1.5}{fig.mediawiki.restructure}

In the 1.4 model, the content was stored in two important tables,
\code{cur} (containing the text and metadata of the current revision
of the page) and \code{old} (containing previous revisions); deleted
pages were kept in \code{archive}. When an edit was made, the
previously current revision was copied to the \code{old} table, and
the new edit was saved to \code{cur}. When a page was renamed, the
page title had to be updated in the metadata of all the \code{old}
revisions, which could be a long operation. When a page was deleted,
its entries in both the \code{cur} and \code{old} tables had to be
copied to the \code{archive} table before being deleted; this meant
moving the text of all revisions, which could be very large and thus
take time.

In the 1.5 model, revision metadata and revision text were split: the
\code{cur} and \code{old} tables were replaced with \code{page} (pages'
metadata), \code{revision} (metadata for all revisions, old or
current) and \code{text} (text of all revisions, old, current or
deleted). Now, when an edit is made, revision metadata don't need to
be copied around tables: inserting a new entry and updating the
\code{page\_latest} pointer is enough. Also, the revision metadata
don't include the page title anymore, only its ID: this removes the
need for renaming all revisions when a page is renamed

The \code{revision} table stores metadata for each revision, but not
their text; instead, they contain a text ID pointing to the
\code{text} table, which contains the actual text. When a page is
deleted, the text of all revisions of the page stays there and doesn't
need to be moved to another table. The \code{text} table is composed
of a mapping of IDs to text blobs; a \code{flags} field indicates if the text
blob is gzipped (for space savings) or if the text blob is only a
pointer to external text storage. Wikimedia sites use a MySQL-backed
external storage cluster with blobs of a few dozen revisions. The
first revision of the blob is stored in full, and following revisions
to the same page are stored as diffs relative to the previous
revision; the blobs are then gzipped. Because the revisions are
grouped per page, they tend to be similar, so the diffs are relatively
small and gzip works well. The compression ratio achieved on Wikimedia
sites nears 98\%.

On the hardware side, MediaWiki has built-in support for load
balancing, added as early as 2004 in MediaWiki 1.2 (when Wikipedia got
its second server---a big deal at the time). The load balancer
(MediaWiki's PHP code that decides which server to connect to) is now
a critical part of Wikimedia's infrastructure, which explains its
influence on some algorithm decisions in the code. The system
administrator can specify, in MediaWiki's configuration, that there is
one master database server and any number of slave database servers;
a weight can be assigned to each server. The load balancer will send
all writes to the master, and will balance reads according to the
weights. It also keeps track of the replication lag of each slave. If
a slave's replication lag exceeds 30 seconds, it will not receive any
read queries to allow it to catch up; if all slaves are lagged more
than 30 seconds, MediaWiki will automatically put itself in read-only
mode.

MediaWiki's ``chronology protector'' ensures that replication lag never
causes a user to see a page that claims an action they've just
performed hasn't happened yet: for instance, if a user renames a page,
another user may still see the old name, but the one who renamed will
always see the new name, because he's the one who renamed it. This is
done by storing the master's position in the user's session if a
request they made resulted in a write query. The next time the user
makes a read request, the load balancer reads this position from the
session, and tries to select a slave that has caught up to that
replication position to serve the request. If none is available, it
will wait until one is. It may appear to other users as though the
action hasn't happened yet, but the chronology remains consistent for
each user.

\end{aosasect1}

\begin{aosasect1}{Requests, Caching and Delivery}

\begin{aosasect2}{Execution Workflow of a Web Request}

\code{index.php} is the main entry point for MediaWiki, and handles
most requests processed by the application servers (i.e., requests that
were not served by the \emph{caching} infrastructure; see below). The
code executed from \code{index.php} performs security checks, loads
default configuration settings from
\code{includes/DefaultSettings.php}, guesses configuration with
\code{includes/Setup.php} and then applies site settings contained in
\code{LocalSettings.php}. Next it instantiates a \code{MediaWiki}
object (\code{\$mediawiki}), and creates a \code{Title} object
(\code{\$wgTitle}) depending on the title and action parameters from
the request.

\code{index.php} can take a variety of action parameters in the URL
request; the default action is \code{view}, which shows the regular
view of an article's content. For example, the request
\url{https://en.wikipedia.org/w/index.php?title=Apple\&action=view}
displays the content of the article ``Apple'' on the English
Wikipedia\footnote{View requests are usually prettified with URL
  rewriting, in this example to
  \url{https://en.wikipedia.org/wiki/Apple}.}. Other frequent actions
include \code{edit} (to open an article for editing), \code{submit}
(to preview or save an article), \code{history} (to show an article's
history) and \code{watch} (to add an article to the user's
watchlist). Administrative actions include \code{delete} (to delete an
article) and \code{protect} (to prevent edits to an article).

\code{MediaWiki::performRequest()} is then called to handle most of
the URL request. It checks for bad titles, read restrictions, local
interwiki redirects, and redirect loops, and determines whether the
request is for a normal or a special page.

Normal page requests are handed over to
\code{MediaWiki::initializeArticle()}, to create an \code{Article}
object for the page (\code{\$wgArticle}), and then to
\code{MediaWiki::performAction()}, which handles ``standard''
actions. Once the action has been completed,
\code{MediaWiki::finalCleanup()} finalizes the request by committing
database transactions, outputting the HTML and launching deferred
updates through the job queue. \code{MediaWiki::restInPeace()} commits
the deferred updates and closes the task gracefully.

If the page requested is a Special page (i.e., not a regular wiki
content page, but a special software-related page such as
\code{Statistics}), \code{SpecialPageFactory::executePath} is called
instead of \code{initializeArticle()}; the corresponding PHP script is
then called. Special pages can do all sorts of magical things, and
each has a specific purpose, usually independent of any one article or
its content. Special pages include various kinds of reports (recent
changes, logs, uncategorized pages) and wiki administration tools
(user blocks, user rights changes), among others. Their execution
workflow depends on their function.

Many functions contain profiling code, which makes it possible to
follow the execution workflow for debugging if profiling is
enabled. Profiling is done by calling the \code{wfProfileIn} and
\code{wfProfileOut} functions to respectively start and stop profiling
a function; both functions take the function's name as a parameter. On
Wikimedia sites, profiling is done for a percentage of all requests,
to preserve performance. MediaWiki sends UDP packets to a central
server that collects them and produces profiling data.

\end{aosasect2}

\begin{aosasect2}{Caching}

MediaWiki itself is improved for performance because it plays a
central role on Wikimedia sites, but it is also part of a larger
operational ecosystem that has influenced its
architecture. Wikimedia's caching infrastructure (structured in
layers) has imposed limitations in MediaWiki; developers worked around
the issues, not by trying to shape Wikimedia's extensively optimized
caching infrastructure around MediaWiki, but rather by making
MediaWiki more flexible, so it could work within that infrastructure
without compromising on performance and caching needs. For example, by
default MediaWiki displays the user's IP in the top-right corner of
the interface (for left-to-right languages) as a reminder that that's
how they're known to the software when they're not logged in. The
\code{\$wgShowIPinHeader} configuration variable allows the system
administrator to disable this feature, thus making the page content
independent of the user: all anonymous visitors can then be served the
exact same version of each page.

The first level of caching (used on Wikimedia sites) consists of
reverse caching proxies (Squids) that intercept and serve most
requests before they make it to the MediaWiki application
servers. Squids contain static versions of entire rendered pages,
served for simple reads to users who aren't logged in to the
site. MediaWiki natively supports Squid and Varnish, and integrates
with this caching layer by, for example, notifying them to purge a
page from the cache when it has been changed. For logged-in users, and
other requests that can't be served by Squids, Squid forwards the
requests to the web server (Apache).

The second level of caching happens when MediaWiki renders and
assembles the page from multiple objects, many of which can be cached
to minimize future calls. Such objects include the page's interface
(sidebar, menus, UI text) and the content proper, parsed from
wikitext. The in-memory object cache has been available in MediaWiki
since the early 1.1 version (2003), and is particularly important to
avoid re-parsing long and complex pages.

Login session data can also be stored in memcached, which lets
sessions work transparently on multiple front-end web servers in a
load-balancing setup (Wikimedia heavily relies on load balancing,
using LVS with PyBal).

Since version 1.16, MediaWiki uses a dedicated object cache for
localized UI text; this was added after noticing that a large part of
the objects cached in memcached consisted of UI messages localized
into the user's language. The system is based on fast fetches of
individual messages from constant databases (CDB), e.g., files with
key-value pairs. CDBs minimize memory overhead and start-up time in
the typical case; they're also used for the interwiki cache.

The last caching layer consists of the PHP opcode cache, commonly
enabled to speed up PHP applications. Compilation can be a lengthy
process; to avoid compiling PHP scripts into opcode every time they're
invoked, a PHP accelerator can be used to store the compiled opcode
and execute it directly without compilation. MediaWiki will ``just
work'' with many accelerators such as APC, PHP accelerator and
eAccelerator.

Because of its Wikimedia bias, MediaWiki is optimized for this
complete, multi-layer, distributed caching
infrastructure. Nonetheless, it also natively supports alternate
setups for smaller sites. For example, it offers an optional
simplistic file caching system that stores the output of fully
rendered pages, like Squid does. Also, MediaWiki's abstract object
caching layer lets it store the cached objects in several places,
including the file system, the database, or the opcode cache.

\end{aosasect2}

\begin{aosasect2}{ResourceLoader}

As in many web applications, MediaWiki's interface has become more
interactive and responsive over the years, mostly through the use of
JavaScript. Usability efforts initiated in 2008, as well as advanced
media handling (e.g., online editing of video files), called for
dedicated front-end performance improvements.

To optimize the delivery of JavaScript and CSS assets, the
ResourceLoader module was developed to optimize delivery of JS and
CSS. Started in 2009, it was completed in 2011 and has been a core
feature of MediaWiki since version 1.17. ResourceLoader works by
loading JS and CSS assets on demand, thus reducing loading and parsing
time when features are unused, for example by older browsers. It also
minifies the code, groups resources to save requests, and can embed
images as data URIs\footnote{For more on ResourceLoader, see
  \url{https://www.mediawiki.org/wiki/ResourceLoader} for the official
  documentation, and the talk \emph{Low Hanging Fruit
    vs. Micro-optimization: Creative Techniques for Loading Web Pages
    Faster} given by Trevor Parscal and Roan Kattouw at OSCON 2011.}.

\end{aosasect2}

\end{aosasect1}

\begin{aosasect1}{Languages}

\begin{aosasect2}{Context and Rationale}

A central part of effectively contributing and disseminating free
knowledge to all is to provide it in as many languages as
possible. Wikipedia is available in more than 280 languages, and
encyclopedia articles in English represent less than 20\% of all
articles. Because Wikipedia and its sister sites exist in so many
languages, it is important not only to provide the content in the
readers' native language, but also to provide a localized interface,
and effective input and conversion tools, so that participants can
contribute content.

For this reason, localization and internationalization (l10n and
i18n) are central components of MediaWiki. The i18n system is
pervasive, and impacts many parts of the software; it's also one of
the most flexible and feature-rich\footnote{For an exhaustive guide to
  internationalization and localization in MediaWiki, see
  \url{https://www.mediawiki.org/wiki/Localisation}.}. Translator
convenience is usually preferred to developer convenience, but this is
believed to be an acceptable cost.

MediaWiki is currently localized in more than 350 languages, including
non-Latin and right-to-left (RTL) languages, with varying levels of
completion. The interface and content can be in different languages,
and have mixed directionality.

\end{aosasect2}

\begin{aosasect2}{Content Language}

MediaWiki originally used per-language encoding, which led to a lot of
issues; for example, foreign scripts could not be used in page
titles. UTF-8 was adopted instead. Support for character sets other
than UTF-8 was dropped in 2005, along with the major database schema
change in MediaWiki 1.5; content must now be encoded in UTF-8.

Characters not available on the editor's keyboard can be customized and
inserted via MediaWiki's Edittools, an interface message that appears
below the edit window; its JavaScript version automatically inserts
the character clicked into the edit window. The WikiEditor extension
for MediaWiki, developed as part of a usability effort, merges special
characters with the edit toolbar. Another extension, called Narayam,
provides additional input methods and key mapping features for
non-ASCII characters.

\end{aosasect2}

\begin{aosasect2}{Interface Language}

Interface messages have been stored in PHP arrays of key-values pairs
since the Phase III software was created. Each message is identified
by a unique key, which is assigned different values across
languages. Keys are determined by developers, who are encouraged to
use prefixes for extensions; for example, message keys for the
UploadWizard extension will start with \code{mwe-upwiz-}, where
\code{mwe} stands for \emph{MediaWiki extension}.

MediaWiki messages can embed parameters provided by the software,
which will often influence the grammar of the message. In order to
support virtually any possible language, MediaWiki's localization
system has been improved and complexified over time to accommodate
languages' specific traits and exceptions, often considered oddities by
English speakers.

For example, adjectives are invariable words in English, but languages
like French require adjective agreement with nouns. If the user
specified their gender in their preferences, the \code{{{GENDER:}}}
switch can be used in interface messages to appropriately address
them. Other switches include \code{{{PLURAL:}}}, for ``simple'' plurals
and languages like Arabic with dual, trial or paucal numbers, and
\code{{{GRAMMAR:}}}, providing grammatical transformation functions
for languages like Finnish whose grammatical cases cause alterations
or inflections.

\end{aosasect2}

\begin{aosasect2}{Localizing Messages}

Localized interface messages for MediaWiki reside in
\code{MessagesXx.php} files, where \code{Xx} is the ISO-639 code of
the language (e.g. \code{MessagesFr.php} for French); default messages
are in English and stored in \code{MessagesEn.php}. MediaWiki
extensions use a similar system, or host all localized messages in an
\code{{\textless}Extension-name{\textgreater}.i18n.php} file. Along
with translations, Message files also include language-dependent
information such as date formats.

Contributing translations used to be done by submitting PHP patches
for the \code{MessagesXx.php} files. In December 2003, MediaWiki 1.1
introduced ``database messages'', a subset of wiki pages in the
MediaWiki namespace containing interface messages. The content of the
wiki page \linebreak
\code{MediaWiki:{\textless}Message-key{\textgreater}} is the
message's text, and overrides its value in the PHP file. Localized
versions of the message are at
\code{MediaWiki:{\textless}Message-key{\textgreater}/{\textless}language-code{\textgreater}};
for example, \linebreak
\code{MediaWiki:Rollbacklink/de}.

This feature has allowed power users to translate (and customize)
interface messages locally on their wiki, but the process doesn't
update i18n files shipping with MediaWiki. In 2006, Niklas Laxstr\"{o}m
created a special, heavily hacked MediaWiki website (now hosted at \linebreak
\url{http://translatewiki.net}) where translators can easily localize
interface messages in all languages simply by editing a wiki
page. The \code{MessagesXx.php} files are then updated in the
MediaWiki code repository, where they can be automatically fetched by
any wiki, and updated using the LocalisationUpdate extension. On
Wikimedia sites, database messages are now only used for
customization, and not for localization any more. MediaWiki extensions
and some related programs, such as bots, are also localized at
translatewiki.net.

To help translators understand the context and meaning of an interface
message, it is considered a good practice in MediaWiki to provide
documentation for every message. This documentation is stored in a
special Message file, with the \code{qqq} language code which doesn't
correspond to a real language. The documentation for each message is
then displayed in the translation interface on
translatewiki.net. Another helpful tool is the \code{qqx} language
code; when used with the \code{\&uselang} parameter to display a
wiki page (e.g.,
\url{https://en.wikipedia.org/wiki/Special:RecentChanges?uselang=qqx}),
MediaWiki will display the message keys instead of their values in the
user interface; this is very useful to identify which message to
translate or change.

Registered users can set their own interface language in their
preferences, to override the site's default interface
language. MediaWiki also supports fallback languages: if a message
isn't available in the chosen language, it will be displayed in the
closest possible language, and not necessarily in English. For
example, the fallback language for Breton is French.

\end{aosasect2}

\end{aosasect1}

\begin{aosasect1}{Users}

Users are represented in the code using instances of the \code{User}
class, which encapsulates all of the user-specific settings (user id,
name, rights, password, email address, etc.). Client classes use
accessors to access these fields; they do all the work of determining
whether the user is logged in, and whether the requested option can be
satisfied from cookies or whether a database query is needed. Most of
the settings needed for rendering normal pages are set in the cookie
to minimize use of the database.

MediaWiki provides a very granular permissions system, with 
a user permission for, basically, every possible action. For example, to perform
the ``Rollback'' action (i.e., to ``quickly rollback the edits of the last
user who edited a particular page''), a user needs the \code{rollback}
permission, included by default in MediaWiki's \code{sysop} user
group. But it can also be added to other user groups, or have a
dedicated user group only providing this permission (this is the case
on the English Wikipedia, with the \code{Rollbackers}
group). Customization of user rights is done by editing the
\code{\$wgGroupPermissions} array in \code{LocalSettings.php}; for
instance, \code{\$wgGroupPermissions['user']['movefile'] = true;}
allows all registered users to rename files. A user can belong to
several groups, and inherits the highest rights associated with each
of them.

However, MediaWiki's user permissions system was really designed with
Wikipedia in mind: a site whose content is accessible to all, and where
only certain actions are restricted to some users. MediaWiki lacks a
unified, pervasive permissions concept; it doesn't provide traditional
CMS features like restricting read or write access by topic or type of
content. A few MediaWiki extensions provide such features to some
extent.

\end{aosasect1}

\begin{aosasect1}{Content}

\begin{aosasect2}{Content Structure}

The concept of namespaces was used in the UseModWiki era of Wikipedia,
where talk pages were at the title ``{\textless}article
name{\textgreater}/Talk''. Namespaces were formally introduced in
Magnus Manske's first ``PHP script''. They were reimplemented a few
times over the years, but have kept the same function: to separate
different kinds of content. They consist of a prefix separated from
the page title by a colon (e.g. \code{Talk:} or \code{File:} and
\code{Template:}); the main content namespace has no prefix. Wikipedia
users quickly adopted them, and they provided the community with
different spaces to evolve. Namespaces have proven to be an important
feature of MediaWiki, as they create the necessary preconditions for a
wiki's community and set up meta-level discussions, community
processes, portals, user profiles, etc.

The default configuration for MediaWiki's main content namespace is to
be flat (no subpages), because it's how Wikipedia works, but it is
trivial to enable subpages. They are enabled in other namespaces
(e.g., \code{User:}, where people can, for instance, work on draft
articles) and display breadcrumbs.

Namespaces separate content by type; within the same namespace, pages
can be organized by topic using categories, a pseudo-hierarchical
organization scheme introduced in MediaWiki 1.3.

\end{aosasect2}

\begin{aosasect2}{Content Processing: MediaWiki Markup Language and Parser}

The user-generated content stored by MediaWiki isn't in HTML, but in a
markup language specific to MediaWiki, sometimes called ``wikitext''. It
allows users to make formatting changes (e.g. bold, italic using
quotes), add links (using square brackets), include templates, insert
context-dependent content (like a date or signature), and make an
incredible number of other magical things happen\footnote{Detailed
  documentation is available at
  \url{https://www.mediawiki.org/wiki/Markup_spec} and the associated
  pages.}.

To display a page, this content needs to be parsed, assembled from all
the external or dynamic pieces it calls, and converted to proper
HTML. The parser is one of the most essential parts of MediaWiki,
which makes it difficult to change or improve. Because hundreds
of millions of wiki pages worldwide depend on the parser to continue
outputting HTML the way it always has, it has to remain extremely
stable.

The markup language wasn't formally specced from the beginning; it
started based on UseModWiki's markup, then morphed and evolved as
needs demanded. In the absence of a formal specification, the
MediaWiki markup language has become a complex and idiosyncratic
language, basically only compatible with MediaWiki's parser; it can't
be represented as a formal grammar. The current parser's specification
is jokingly referred to as ``whatever the parser spits out from
wikitext, plus a few hundred test cases''.

There have been many attempts at alternative parsers, but none has
succeeded so far. In 2004 an experimental tokenizer was written by
Jens Frank to parse wikitext, and enabled on Wikipedia; it had to be
disabled three days later because of the poor performance of PHP
array memory allocations. Since then, most of the parsing has been
done with a huge pile of regular expressions, and a ton of helper
functions. The wiki markup, and all the special cases the parser needs
to support, have also become considerably more complex, making future
attempts even more difficult.

A notable improvement was Tim Starling's preprocessor rewrite in
MediaWiki 1.12, whose main motivation was to improve the parsing
performance on pages with complex templates. The preprocessor converts
wikitext to an XML DOM tree representing parts of the document
(template invocations, parser functions, tag hooks, section headings,
and a few other structures), but can skip ``dead branches'', 
such as unfollowed \code{\#switch} cases and unused defaults
for template arguments, in template expansion. The parser then iterates through the DOM
structure and converts its content to HTML.

Recent work on a visual editor for MediaWiki has made it necessary to
improve the parsing process (and make it faster), so work has resumed
on the parser and intermediate layers between MediaWiki markup and
final HTML (see \emph{Future}, below).

\end{aosasect2}

\begin{aosasect2}{Magic Words and Templates}

MediaWiki offers ``magic words'' that modify the general behavior of the
page or include dynamic content into it. They consist of: behavior
switches like \code{\_\_NOTOC\_\_} (to hide the automatic table of
content) or \code{\_\_NOINDEX\_\_} (to tell search engines not to index
the page); variables like \code{{{CURRENTTIME}}} or
\code{{{SITENAME}}}; and parser functions, i.e., magic words that can
take parameters, like \code{{{lc:{\textless}string{\textgreater}}}}
(to output \code{{\textless}string{\textgreater}} in
lowercase). Constructs like \code{{{GENDER:}}}, \code{{{PLURAL:}}} and
\code{{{GRAMMAR:}}}, used to localize the UI, are parser functions.

The most common way to include content from other pages in a MediaWiki
page is to use templates. Templates were really intended to be used to
include the same content on different pages, e.g., navigation panels or
maintenance banners on Wikipedia articles; having the ability to
create partial page layouts and reuse them in thousands of articles
with central maintenance made a huge impact on sites like Wikipedia.

However, templates have also been used (and abused) by users for a
completely different purpose. MediaWiki 1.3 made it possible for
templates to take parameters that change their output; the ability to
add a default parameter (introduced in MediaWiki 1.6) enabled the
construction of a functional programming language implemented on top
of PHP, which was ultimately one of the most costly features in terms
of performance.

Tim Starling then developed additional parser functions (the
ParserFunctions extension), as a stopgap measure against insane
constructs created by Wikipedia users with templates. This set of
functions included logical structures like \code{\#if} and
\code{\#switch}, and other functions like \code{\#expr} (to evaluate
mathematical expressions) and \code{\#time} (for time formatting).

Soon enough, Wikipedia users started to create even more complex
templates using the new functions, which considerably degraded the
parsing performance on template-heavy pages. The new preprocessor
introduced in MediaWiki 1.12 (a major architectural change) was
implemented to partly remedy this issue. Recently, MediaWiki
developers have discussed the possibility of using an actual scripting
language, perhaps Lua, to improve performance.

\end{aosasect2}

\begin{aosasect2}{Media Files}

Users upload files through the \code{Special:Upload} page;
administrators can configure the allowed file types through an
extension whitelist. Once uploaded, files are stored in a folder on
the file system, and thumbnails in a dedicated \code{thumb} directory.

Because of Wikimedia's educational mission, MediaWiki supports file
types that may be uncommon in other web applications or CMSes, like
SVG vector images, and multipage PDFs and DjVus. They are rendered
as PNG files, and can be thumbnailed and displayed inline, as are more
common image files like GIFs, JPGs and PNGs.

When a file is uploaded, it is assigned a \code{File:} page containing
information entered by the uploader; this is free text and usually
includes copyright information (author, license) and items describing
or classifying the content of the file (description, location, date,
categories, etc.). While private wikis may not care much about this
information, on media libraries like Wikimedia Commons it are
critical to organise the collection and ensure the legality of sharing
these files. It has been argued that most of these metadata should, in
fact, be stored in a queryable structure like a database table. This
would considerably facilitate search, but also attribution and reuse
by third parties---for example, through the API.

Most Wikimedia sites also allow ``local'' uploads to each wiki, but the
community tries to store freely licensed media files in Wikimedia's
free media library, Wikimedia Commons. Any Wikimedia site can display
a file hosted on Commons as if it were hosted locally. This custom
avoids having to upload a file to every wiki to use it there.

As a consequence, MediaWiki natively supports foreign media
repositories, i.e., the ability to access media files hosted on
another wiki through its API and the \code{ForeignAPIRepo}
system. Since version 1.16, any MediaWiki website can easily use files
from Wikimedia Commons through the \code{InstantCommons} feature. When
using a foreign repository, thumbnails are stored locally to save
bandwidth. However, it is not (yet) possible to upload to a foreign
media repository from another wiki.

\end{aosasect2}

\end{aosasect1}

\begin{aosasect1}{Customizing and Extending MediaWiki}

\begin{aosasect2}{Levels}

MediaWiki's architecture provides different ways to customize and
extend the software. This can be done at different levels of access:

\begin{aosaitemize}

\item System administrators can install extensions and skins, and
  configure the wiki's separate helper programs (e.g., for image
  thumbnailing and TeX rendering) and global settings (see
  \emph{Configuration} above).

\item Wiki sysops (sometimes called ``administrators'' too) can edit
  site-wide gadgets, JavaScript and CSS settings.

\item Any registered user can customize their own experience and
  interface using their preferences (for existing settings, skins and
  gadgets) or make their own modifications (using their personal JS
  and CSS pages).

\end{aosaitemize}

External programs can also communicate with MediaWiki through its
machine API, if it's enabled, basically making any feature and data
accessible to the user.

\end{aosasect2}

\begin{aosasect2}{JavaScript and CSS}

MediaWiki can read and apply site-wide or skin-wide JavaScript and CSS
using custom wiki pages; these pages are in the \code{MediaWiki:}
namespace, and thus can only be edited by sysops; for example,
JavaScript modifications from \code{MediaWiki:Common.js} apply to all
skins, CSS from \code{MediaWiki:Common.css} applies to all skins, but
\code{MediaWiki:Vector.css} only applies to users with the Vector
skin.

Users can do the same types of changes, which will only apply to their
own interface, by editing subpages of their user page
(e.g. \code{User:{\textless}Username{\textgreater}/common.js} for
JavaScript on all skins,
\code{User:{\textless}Username{\textgreater}/common.css} for CSS on
all skins, or \code{User:{\textless}Username{\textgreater}/vector.css}
for CSS modifications that only apply to the Vector skin).

If the Gadgets extension is installed, sysops can also edit gadgets,
i.e., snippets of JavaScript code, providing features that can be turned
on and off by users in their preferences. Upcoming developments on
gadgets will make it possible to share gadgets across wikis, thus
avoiding duplication.

This set of tools has had a huge impact and greatly increased the
democratization of MediaWiki's software development. Individual users
are empowered to add features for themselves; power users can share
them with others, both informally and through globally configurable
sysop-controlled systems. This framework is ideal for small,
self-contained modifications, and presents a lower barrier to entry
than heavier code modifications done through hooks and extensions.

\end{aosasect2}

\begin{aosasect2}{Extensions and Skins}

When JavaScript and CSS modifications are not enough, MediaWiki
provides a system of hooks that let third-party developers run custom
PHP code before, after, or instead of MediaWiki code for particular
events\footnote{MediaWiki hooks are referenced at
  \url{https://www.mediawiki.org/wiki/Manual:Hooks}.}. MediaWiki
extensions use hooks to plug into the code.

Before hooks existed in MediaWiki, adding custom PHP code meant
modifying the core code, which was neither easy nor recommended. The
first hooks were proposed and added in 2004 by Evan Prodromou; many
more have been added over the years when needed. Using hooks, it is
even possible to extend MediaWiki's wiki markup with additional
capabilities using tag extensions.

The extension system isn't perfect; extension registration is based on
code execution at startup, rather than cacheable data, which limits
abstraction and optimization and hurts MediaWiki's performance. But
overall, the extension architecture is now a fairly flexible
infrastructure that has helped make specialized code more modular,
keeping the core software from expanding (too) much, and making it
easier for third-party users to build custom functionality on top of
MediaWiki.

Conversely, it's very difficult to write a new skin for MediaWiki
without reinventing the wheel. In MediaWiki, skins are PHP classes
each extending the parent \code{Skin} class; they contain functions
that gather the information needed to generate the HTML. The
long-lived ``MonoBook'' skin was difficult to customize because it
contained a lot of browser-specific CSS to support old browsers;
editing the template or CSS required many subsequent changes to
reflect the change for all browsers and platforms.

\end{aosasect2}

\begin{aosasect2}{API}

The other main entry point for MediaWiki, besides \code{index.php}, is
\code{api.php}, used to access its machine-readable web query API
(Application Programming Interface).

Wikipedia users originally created ``bots'' that worked by screen
scraping the HTML content served by MediaWiki; this method was very
unreliable and broke many times. To improve this situation, developers
introduced a read-only interface (located at \code{query.php}), which
then evolved into a full-fledged read and write machine API providing
direct, high-level access to the data contained in the MediaWiki
database\footnote{Exhaustive documentation of the API is available at
  \url{https://www.mediawiki.org/wiki/API}.}.

Client programs can use the API to login, get data, and post
changes. The API supports thin web-based JavaScript clients and
end-user applications. Almost anything that can be done via the web
interface can basically be done through the API. Client libraries
implementing the MediaWiki API are available in many languages,
including Python and .NET.

\end{aosasect2}

\end{aosasect1}

\begin{aosasect1}{Future}

What started as a summer project done by a single volunteer PHP
developer has grown into MediaWiki, a mature, stable wiki engine
powering a top-ten website with a ridiculously small operational
infrastructure. This has been made possible by constant optimization
for performance, iterative architectural changes and a team of awesome
developers.

The evolution of web technologies, and the growth of Wikipedia, call
for ongoing improvements and new features, some of which require major
changes to MediaWiki's architecture. This is, for example, the case
for the ongoing visual editor project, which has prompted renewed work
on the parser and on the wiki markup language, the DOM and final HTML
conversion.

MediaWiki is a tool used for very different purposes. Within Wikimedia
projects, for instance, it's used to create and curate an encyclopedia
(Wikipedia), to power a huge media library (Wikimedia Commons), to
transcribe scanned reference texts (Wikisource), and so on. In other
contexts, MediaWiki is used as a corporate CMS, or as a data
repository, sometimes combined with a semantic framework. These
specialized uses that weren't planned for will probably continue to
drive constant adjustments to the software's internal structure. As
such, MediaWiki's architecture is very much alive, just like the
immense community of users it supports.

\end{aosasect1}

\begin{aosasect1}{Further Reading}

\begin{aosaitemize}

\item MediaWiki documentation and support:
  \url{https://www.mediawiki.org}.

\item Automatically generated MediaWiki documentation:
  \url{http://svn.wikimedia.org/doc/}.

\item Domas Mituzas, \emph{Wikipedia: site internals, configuration,
  code examples and management issues}, MySQL Users conference,
  2007. Full text available at \url{http://dom.as/talks/}.

\end{aosaitemize}

\end{aosasect1}

\begin{aosasect1}{Acknowledgments}

This chapter was created collaboratively. Guillaume Paumier wrote most
of the content by organizing the input provided by MediaWiki users and
core developers. Sumana Harihareswara coordinated the interviews and
input-gathering phases. Many thanks to Antoine Musso, Brion Vibber,
Chad Horohoe, Tim Starling, Roan Kattouw, Sam Reed, Siebrand Mazeland,
Erik M\"{o}ller, Magnus Manske, Rob Lanphier, Amir Aharoni, Federico Leva,
Graham Pearce and others for providing input and/or reviewing the
content.

\end{aosasect1}

\end{aosachapter}
